\section{Übersetzung -- Media Use Questionnaire}\label{chap.appendix_mediaUseQuestionnaire}

%Media Use Questionnaire
Für die Erfassung des \textit{Media Multitasking Index (MMI)} von \citet{Ophir2009} wurde der aus dem englisch sprachigen Raum stammende Fragebogen \textit{Media Use Questionnaire} aus dem Englischen ins Deutsche übersetzt.
Die Bewertung der Medien-Multitasking-Matrix (siehe \ref{section.erhebungsinstrumente}) erfolgt mittels folgender Skala: 1 = meistens (engl.: most of the time); 0.67 = etwas (engl.: some of the time); 0.33 = wenig (engl.: a little of the time); 0 = nie (engl.: never).  \\

%Tabelle
\begin{center}
    \begin{longtable}[t]{|l|p{6.6 cm}|p{6.6 cm}|}
    \caption{Übersetzung Medien -- Media Use Questionnaire} \\ \hline
        \textbf{Nr.} & \textbf{Englisch} & \textbf{Deutsch} \\ \hline
        \endfirsthead
        \hline
        \textbf{Nr.} & \textbf{Englisch} & \textbf{Deutsch} \\ \hline
        \endhead 
        & \multicolumn{2}{c|}{Fortsetzung auf der nächsten Seite $...$ } \\ \hline
        \endfoot
        \hline
        \endlastfoot
        1 & print media & Druckmedien \\
        2 & television & Fernsehen \\
        3 & computer-based video (such as YouTube or online television episodes) & Online Video (wie zum Beispiel Youtube) \\
        4 & music & Musik \\
        5 & nonmusic audio & Nicht-musikalische Audiomedien (z.B. Hörbücher, Podcast, etc.) \\
        6 & video or computer games & Video oder Computer Games \\
        7 & telephone and mobile phone voice calls & Telefonieren (Mobil- und / oder Festnetz) \\
        8 & instant messaging & Instant Messaging (z.B. Skype, Windows Live Messenger, Yahoo Messenger, etc.) \\
        9 & SMS (text messaging) & SMS (Textnachrichten) \\
        10 & email & Email \\
        11 & web surfing & Internet-Surfen \\
        12 & other computer-based application  & Andere Computerbasierte-Tätigkeiten (z.B. Word, Videobearbeitung, programmieren, etc.) \\
        \end{longtable}
	\label{tab.muqUebersetzung}
\end{center}

