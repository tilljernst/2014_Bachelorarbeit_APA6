\section{Fragebogen}\label{chap.appendix_fragebogen}
Durch die Übernahme des Umfragebogens aus dem Internet, wurde dieser für die Darstellung in diesem Dokument im Folgenden leicht angepasst. Dies betrifft insbesondere die Eingabefelder und Auswahllisten, die im Onlinefragebogen zur Verfügung gestanden haben. Diese Elemente werden in diesem Dokument bei den entsprechenden Fragen mittels Vermerk angedeutet.
%Startseite
\subsection{Startseite}\label{anhangSesction.Startseite}
\subsubsection{Herzlich Willkommen}
Vielen Dank, dass Du Dir Zeit nimmst, den folgenden Fragebogen zu beantworten. Das Ausfüllen des Fragebogens wird ca. 10 bis 15 Minuten in Anspruch nehmen.\par
Bei den Fragen kommt es mir auf deine subjektiven Einschätzungen an, d.h. es gibt keine richtigen oder falschen Antworten. Mich interessiert deine persönliche Meinung. Bitte lies jeweils genau die Instruktionen zu den einzelnen Fragen und beantworte alle Fragen zügig und vertraue dabei deinem spontanen Urteil. Wenn dennoch eine Aussage für Dich schwierig einzuschätzen ist, versuche diese bitte trotzdem zu beantworten. \par
Sämtliche Angaben werden streng vertraulich behandelt. Daten werden nicht an dritte Personen weiter gegeben. \par
Besten Dank für deine Mithilfe.\par
Till J. Ernst\\
Mailto: ernsttil(at)students.zhaw.ch
\subsubsection{Ergebnisse}
Die Ergebnisse der Studie werden anonymisiert, ausgewertet und in einer Bachelorarbeit publiziert. Rückschlüsse auf einzelne Teilnehmende der Befragung sind nicht möglich.\par
Falls Du an den Ergebnissen interessiert bist, hast Du am Schluss die Möglichkeit deine Mailadresse zu hinterlegen, damit ich Dir die Ergebnisse persönlich zusenden kann. \par
Die Ergebnisse werden Ende Frühlingssemester 2014 in Form einer Bachelorarbeit im Netz publiziert (\url{http://www.psychologie.zhaw.ch/de/psychologie/studium/master-undbachelorarbeiten.html}).
\subsubsection{Wettbewerb}
Am Ende von diesem Fragebogen hast Du die Möglichkeit an einem Wettbewerb teilzunehmen. Du kannst dabei Gutscheine beim Athleticum Sportmarkets gewinnen. \par
Um am Wettbewerb teilzunehmen wirst Du am Schluss gebeten deine Mailadresse zu hinterlegen. Diese Mailadresse dient nur zur Teilnahme am Wettbewerb. Ein Rückschluss auf die Antworten ist damit nicht möglich. Über den Wettbewerb wird keine Korrespondenz geführt.
\begin{figure}[h]
     \centering
     \includegraphics[scale=0.6]{images/anhang/sponsoring_athleticum_mittel.jpg}
\end{figure}
%Demographische Daten
\subsection{Demographische Daten}\label{anhangSesction.demograpData}
\subsubsection{Angaben zu deiner Person}
Gleich zu Beginn möchte ich Dich um einige Angaben zu deiner Person bitten.
\paragraph{Geschlecht (Auswahl)}
    \begin{itemize}
      \item weiblich
      \item männlich
      \item keine Angabe
    \end{itemize}

\paragraph{Wie alt bis Du? (Eingabefeld)}
Bitte das aktuelle Altersjahr in Zahlen angeben.
\paragraph{Wo studierst Du? (Auswahl)}
    \begin{itemize}
      \item Universität
      \item Fachhochschule
    \end{itemize}
\paragraph{Studienrichtung (Eingabefeld)}
    \begin{itemize}
      \item Hauptfach
      \item Nebenfach (optional)
    \end{itemize}
\paragraph{Studentenstatus (Auswahl)}
    \begin{itemize}
      \item Vollzeit
      \item Teilzeit
    \end{itemize}
\paragraph{Gehst Du neben dem Studium einer bezahlten Arbeit nach? (Auswahl und Eingabefeld)}
Falls Du berufstätig bist, fülle bitte die durchschnittliche Anzahl Stunden ein, die Du in einer Woche arbeitest.
    \begin{itemize}
      \item Ja (inkl. Eingabefeld)
      \item Nein
    \end{itemize}
\paragraph{Zivilstand (Auswahl)}
    \begin{itemize}
      \item ledig
      \item verheiratet
      \item getrennt
      \item geschieden
      \item verwitwet
    \end{itemize}
\paragraph{Hast Du Kinder? (Auswahl)} 
    \begin{itemize}
      \item Ja
      \item Nein
    \end{itemize}    
%Medie Use Questionnaire
\subsection{MUQ--Media Use Questionnaire}\label{anhangSection.muq}   
\subsubsection{Mediennutzung}    
In den folgenden Fragen möchte ich gerne von Dir wissen, welche Medien Du wie oft einsetzt und welche Du gleichzeitig verwendest.
\paragraph{Wieviele Minuten pro Tag nutzt Du die folgenden 12 Medienformen durchschnittlich (privat, Studium und geschäftlich)? (Eingabefeld)}
Bitte fülle die Anzahl Minuten in ganzen Zahlen ein (Beispiel: Druckmedien = 60; Fernsehen = 45).
    \begin{itemize}
      \item Druckmedien (z.B. Bücher, Zeitungen, Zeitschriften, etc.)
      \item Fernsehen
      \item Online Video (wie zum Beispiel Youtube)
      \item Musik
      \item Nicht-Musikalische Audiomedien (wie Hörbücher, Podcast, etc.)
      \item Video oder Computer Games
      \item Telefonieren (Mobile- und / oder Festnetz)
      \item Instant Messaging (wie zum Beispiel Skype, Windows Live Messenger, Yahoo Messenger, etc.)
      \item SMS (Textnachrichten)
      \item Email
      \item Internet-Surfen
      \item Andere computerbasierte Tätigkeiten (z.B. Textverarbeitung, Videobearbeitung, programmieren, etc.)
    \end{itemize} 
\paragraph{Bitte trage ein, wie oft Du ein Medium gleichzeitig mit einem anderen verwendest. Ein Beispiel könnte sein, wenn Du ein Buch liest und gleichzeitig Musik hörst (Auswahl mittels Matrix)}
Auf der linken Spalte befindet sich das Medium, dem Du Dich hauptsächlich widmest. Auf der horizontalen Zeile befinden sich die Medien, die Du zusammen mit dem Medium auf der linken Seite verwendest.\par
Verwende für deine Zuteilung die Wertung mittels Dropdown-Liste (meistens, etwas, wenig, nie). Machst Du keine Angaben zu einem Nebenmedium, gilt dies als 'nie' (Standard).\par
\textbf{Probleme mit der Ansicht:} Falls die Darstellung einer Liste gleicht, kann dies daran liegen, dass das Fenster deines Browsers zu klein ist. Bitte vergrössere das Fenster, damit Du alle Medien auf einer Zeile siehst oder falls dies nicht geht, gehe bitte der Liste nach vor. Die fett markierten Medien sind die Hauptmedien, die folgenden normal gedruckten Medien sind diejenigen, die Du gleichzeitig nutzt.\par
    \begin{itemize}
        \item \textbf{Druckmedien mit:}\\
Fernsehen, Online Video, Musik, Nicht-musikalische Audiomedien, Video oder Computergames, Telefonieren, Instant Messaging, SMS, Emails, Surfen, andere computer-basierte Tätigkeiten.
        \item \textbf{Fernsehen mit:}\\
Druckmedien, Online Video, Musik, Nicht-musikalische Audiomedien, Video oder Computergames, Telefonieren, Instant Messaging, SMS, Emails, Surfen, andere computer-basierte Tätigkeiten.
        \item \textbf{Online Video mit:}\\
Druckmedien, Fernsehen, Musik, Nicht-musikalische Audiomedien, Video oder Computergames, Telefonieren, Instant Messaging, SMS, Emails, Surfen, andere computer-basierte Tätigkeiten. 
        \item \textbf{Musik mit:}\\
Druckmedien, Fernsehen, Online Video, Nicht-musikalische Audiomedien, Video oder Computergames, Telefonieren, Instant Messaging, SMS, Emails, Surfen, andere computer-basierte Tätigkeiten. 
        \item \textbf{Nicht-musikalische Audiomedien mit:}\\
Druckmedien, Fernsehen, Online Video, Musik, Video oder Computergames, Telefonieren, Instant Messaging, SMS, Emails, Surfen, andere computer-basierte Tätigkeiten. 
        \item \textbf{Video oder Computergames mit:}\\
Druckmedien, Fernsehen, Online Video, Musik, Nicht-musikalische Audiomedien, Telefonieren, Instant Messaging, SMS, Emails, Surfen, andere computer-basierte Tätigkeiten. 
        \item \textbf{Telefonieren mit:}\\
Druckmedien, Fernsehen, Online Video, Musik, Nicht-musikalische Audiomedien, Video oder Computergames, Instant Messaging, SMS, Emails, Surfen, andere computer-basierte Tätigkeiten. 
        \item \textbf{Instant Messaging mit:}\\
Druckmedien, Fernsehen, Online Video, Musik, Nicht-musikalische Audiomedien, Video oder Computergames, Telefonieren, SMS, Emails, Surfen, andere computer-basierte Tätigkeiten. 
        \item \textbf{SMS mit:}\\
Druckmedien, Fernsehen, Online Video, Musik, Nicht-musikalische Audiomedien, Video oder Computergames, Telefonieren, Instant Messaging, Emails, Surfen, andere computer-basierte Tätigkeiten. 
        \item \textbf{Emails mit:}\\
Druckmedien, Fernsehen, Online Video, Musik, Nicht-musikalische Audiomedien, Video oder Computergames, Telefonieren, Instant Messaging, SMS, Surfen, andere computer-basierte Tätigkeiten. 
        \item \textbf{Surfen mit:}\\
Druckmedien, Fernsehen, Online Video, Musik, Nicht-musikalische Audiomedien, Video oder Computergames, Telefonieren, Instant Messaging, SMS, Emails, andere computer-basierte Tätigkeiten.
        \item \textbf{andere computer-basierte Tätigkeiten mit:}\\
Druckmedien, Fernsehen, Online Video, Musik, Nicht-musikalische Audiomedien, Video oder Computergames, Telefonieren, Instant Messaging, SMS, Emails, Surfen.        
    \end{itemize}
%Attentional Control Scale
\subsection{ACS -- Attentional Control Scale}\label{anhangSection.macs}   
\subsubsection{Aufmerksamkeit}  
In den folgenden Fragen geht es um deine Aufmerksamkeitskontrolle.
\paragraph{Inwieweit kannst Du den folgenden Aussagen zustimmen? (Auswahl)}
Bitte beantworte die folgenden Fragen gemäss deiner Zustimmung (1 = fast nie, 2 = manchmal, 3 = oft und 4 = immer).
    \begin{itemize}
      \item In einer lauten Umgebung habe ich Mühe, mich auf eine schwierige Aufgabe zu konzentrieren.
      \item Wenn mich auf die Lösung eines Problems konzentrieren muss, habe ich Mühe meine Aufmerksamkeit zu fokussieren.
      \item Wenn ich mich intensiv mit etwas beschäftige, werde ich durch Ereignisse um mich herum abgelenkt.
      \item Musik im gleichen Raum stört meine Konzentration nicht.
      \item Wenn ich mich auf etwas konzentriere, so kann ich meine Aufmerksamkeit derart fokussieren, dass ich alles um mich herum vergesse.
      \item Wenn ich lesen oder etwas lernen muss, lasse ich mich leicht ablenken, wenn Leute im selben Raum miteinander sprechen.
      \item Wenn ich meine Aufmerksamkeit auf etwas zu konzentrieren versuche, so habe ich Schwierigkeiten ablenkende Gedanken abzublocken.
      \item Ich habe Mühe, mich zu konzentrieren, wenn ich aufgeregt bin.
      \item Wenn ich mich konzentriere vergesse ich, dass ich durstig oder hungrig bin.
      \item Ich kann rasch von einer Aufgabe zur nächsten Aufgabe wechseln.
      \item Es dauert eine Weile, bis ich mich in eine neue Aufgabe eingearbeitet habe.
      \item Es bereitet mir Schwierigkeiten, meine Aufmerksamkeit zwischen dem Zuhören und dem Niederschreiben von Informationen während der Vorlesung zu koordinieren.
      \item Wenn erforderlich, kann ich mich leicht für ein neues Thema begeistern.
      \item Es fällt mir leicht, während eines Telefonats gleichzeitig zu lesen oder zu schreiben.
      \item Ich habe Mühe, zwei verschiedene Gespräche gleichzeitig zu führen.
      \item Es fällt mir schwer, spontan auf neue Ideen zu kommen.
      \item Nachdem ich abgelenkt oder unterbrochen wurde, kann ich meine Aufmerksamkeit mühelos auf die vorherige Arbeit lenken.
      \item Wenn mir ein störender Gedanke in den Sinn kommt, so kann ich ihn ohne grosse Mühe wieder vergessen.
      \item Es fällt mir leicht, zwischen zwei unterschiedlichen Tätigkeiten hin und her zu wechseln.
      \item Es fällt mir schwer, mich von meiner bisherigen Denkweise zu lösen und einen Sachverhalt von einer anderen Seite zu betrachten.
    \end{itemize}
%Attentional Control Scale
\subsection{FS -- Flourishing Scale}\label{anhangSection.fs}   
\subsubsection{Wohlbefinden} 
In den folgenden Fragen geht es um dein persönliches Wohlbefinden.
\paragraph{Unten findest Du acht Aussagen, denen Du zustimmen oder widersprechen kannst. (Auswahl)}
Bewerte jede Aussage auf einer Skala mittels 'trifft überhaupt nicht zu', 'trifft nicht zu', 'trifft kaum zu', 'gemischt oder weder Zustimmung noch Ablehnung', 'trifft etwas zu', 'trifft zu' und 'trifft voll und ganz zu').
    \begin{itemize}
      \item Ich führe ein zielgerichtetes und sinnvolles Leben.
      \item Meine sozialen Beziehungen stärken und bereichern mich.
      \item Ich bin engagiert und interessiere mich für meine täglichen Aktivitäten.
      \item Ich trage aktiv zum Glück und zum Wohlbefinden anderer bei.
      \item In den Aktivitäten, die mir wichtig sind, bin ich kompetent und fähig.
      \item Ich bin ein guter Mensch und ich führe ein gutes Leben.
      \item Ich bin optimistisch, was meine Zukunft betrifft.
      \item Die Leute respektieren mich.
    \end{itemize}  
%SPANE
\subsection{SPANE -- Scale of Pos and Neg Experience}\label{anhangSection.spane}   
\subsubsection{Positive und negative Erfahrungen} 
In den folgenden Fragen geht es um deine persönlichen Erfahrungen der letzten vier Wochen.
\paragraph{Erinnere dich daran, was Du in den letzten vier Wochen gemacht hast und wie Du Dich dabei gefühlt hast. Anschliessend entscheide Dich anhand der unten stehenden Skala, wie oft Du diese Gefühle und Emotionen erlebt hast. (Auswahl)}
Bewerte jede Aussage auf einer Skala mittels 'sehr selten oder nie', 'selten', 'manchmal', 'oft' und 'sehr oft oder immer').
    \begin{itemize}
      \item positiv
      \item negativ
      \item gut
      \item schlecht
      \item angenehm
      \item unangenehm
      \item glücklich
      \item traurig      
      \item ängstlich
      \item froh
      \item wütend
      \item zufrieden
    \end{itemize}  
%Entscheidung Wettbewerb
\subsection{Entscheidung Wettbewerb}\label{anhangSection.wettbewerb}   
Du bist nun fast am Ende dieser Umfrage angelangt!\par
Wenn Du möchtest, kannst Du nun am Wettbewerb teilnehmen und Dich für die Ergebnisse dieser Umfrage anmelden. Dazu bitte ich Dich unten die entsprechende Wahl zu treffen. \par
Die Ergebnisse dieser Bachelorarbeit werden voraussichtlich Ende Juni 2014 publiziert. Falls Du Dich für die Ergebnisse interessierst, werde ich Dich in diesem Zeitraum per Mail kontaktieren.\par
\textbf{Drücke bitte in jedem Fall auf den 'Weiter' Button! Danke!}
\paragraph{Teilnahme Wettbewerb (Auswahl und Eingabefeld)}
Wenn Du am Wettbewerb teilnehmen möchtest, gib bitte deine Mailadresse an.
    \begin{itemize}
      \item Ich möchte am Wettbewerb teil nehmen.
      \item Nein Danke!      
    \end{itemize}
\paragraph{Ergebnisse zu dieser Umfrage (Auswahl und Eingabefeld)}
Wenn Du Information zu dieser Umfrage möchtest, gib bitte deine Mailadresse an.
    \begin{itemize}
      \item Information zu den Ergebnissen.
      \item Nein Danke!      
    \end{itemize}
Zu gewinnen gibt es Gutscheine im Athleticum Sport Market im Wert von 2 x CHF 100.- und 4 x CHF 50.-, die in jeder Athleticum Filiale eingelöst werden können.
%Endseite
\subsection{Endseite}\label{anhangSection.endseite}
\textbf{Gratuliere!}\par
Nun bist Du am eigentlichen Ende dieser Umfrage angelangt. Nochmals vielen herzlichen Dank!
\begin{figure}
     \centering
     \includegraphics[scale=0.6]{images/anhang/ende.jpeg}
\end{figure}    
    